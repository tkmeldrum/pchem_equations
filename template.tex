\documentclass[12pt]{article}
\usepackage{multicol}
\usepackage{calc}
\usepackage[letterpaper,portrait]{geometry}
\usepackage[fleqn]{mathtools}
\usepackage{amsmath,amsthm,amsfonts,amssymb}
\usepackage[normalem]{ulem}
\usepackage{hyperref}
\usepackage{siunitx,booktabs,cancel,caption,cleveref,colortbl,csquotes,helvet,mathpazo,multirow,listings,pgfplots,xcolor}

\geometry{top=0.5in,left=.5in,right=.5in,bottom=0.5in}

% Turn off header and footer
\pagestyle{empty}

% Don't print section numbers
\setcounter{secnumdepth}{0}

% Include some SI units that aren't normally present
\DeclareSIUnit[number-unit-product = {}]{\centimeter}{cm}
\DeclareSIUnit[number-unit-product = {}]{\gram}{g}
\DeclareSIUnit[number-unit-product = {}]{\atm}{atm}
\DeclareSIUnit[number-unit-product = {}]{\cal}{cal}

\setlength{\mathindent}{0cm}

\makeatletter
\def\input@path{{/path/to/latex/files/}}
\makeatother

\begin{document}
\raggedcolumns

% adjust the number of columns as needed
\begin{multicols*}{2}

%for each subsection, add a heading (e.g., \subsection) and input the corresponding file
\subsection{Constants and units}
\begin{align*}
c&= \SI{2.998e8}{\meter\per\second}\\
h&= \SI{6.626e-34}{\joule\second}\\
\hbar &= \frac{h}{2 \pi } = \SI{1.055e-34}{\joule\second}\\
N_A&= \SI{6.022e23}{\per\mole}\\
k_B&= \SI{1.381e-23}{\joule\per\kelvin}\\
%&= \SI{1.381e-16}{\gram\centi\meter\squared\per\second\squared\per\kelvin}\\
R&= \SI{8.3145}{\joule\per\kelvin\per\mole}\\
&= \SI{0.08206}{\liter\atm\per\mole\per\kelvin}\\
%&= \SI{0.083145}{\liter\bar\per\mole\per\kelvin}\\
m_{e^-}&=\SI{9.109e-31}{\kilo\gram}\\
e &= \SI{1.602e-19}{\coulomb}\\
m_{p^+}&=\SI{1.673e-27}{\kilo\gram}\\
m_{n^0} &= \SI{1.675e-27}{\kilo\gram}\\
a_0 &= \SI{0.5292e-10}{\meter}\\
\varepsilon_0 &= \SI{8.8542e-12}{\coulomb\squared\per\newton\per\meter\squared}\\
R_H&=\SI{1.097e5}{\per\centi\meter}\\
\SI{1}{\newton} &= \SI{1}{\kilo\gram\meter\per\second\squared}\\
\SI{1}{\joule}&= \SI{1}{\newton\meter} = \SI{1}{\kilo\gram\meter\squared\per\second\squared}\\
\SI{1}{\eV}&=\SI{1.602e-19}{\joule}\\
\SI{1}{\per\centi\meter}&=\SI{1.986e-23}{\joule}\\
\SI{1}{\amu} &= \SI{1.661e-27}{\kilo\gram}\\
\end{align*}

%\SI{1000}{\liter} &= \SI{1}{\meter\cubed}\\

\subsection{Thermodynamics}
\begin{align*}
%\delta w&=-p_{ext}dV\\
\Delta U &= w + q\\
w&=-p_{ext}\Delta V\\
w_{\textrm{iso-T}}&=-\int_{V_1}^{V_2}\frac{nRT}{V}dV=-nRT\ln\frac{V_2}{V_1}\\
H &= U + pV\\
\Delta H &= q_p = \int dH=\int C_p(T)dT\\
C_V&=\left(\frac{\partial U}{\partial T}\right)_V\\
C_p&=\left(\frac{\partial H}{\partial T}\right)_p\\
C_p&=C_V+nR\\
%&\textcolor{blue}{C_{p,\textrm{low-T}}\propto T^3; S\left(T=T_{\textrm{low}}\right)=C_p/3}\\
%&U_{\textrm{monatomic ideal gas}}=\frac{3}{2}nRT\\
\Delta S &= \frac{q_{rev}}{T}= nR\ln \frac{V_2}{V_1} + C_V \ln \frac{T_2}{T_1}\\
\Delta S &= \int \frac{C_p}{T} dT + \sum_i\frac{\Delta_{trs} H}{T}\\
%\ulcorner \frac{S}{R}&=\frac{7}{2}+\ln {\left[ \left( \frac{2\pi m k_B T}{h^2}\right)^{3/2}\frac{V}{N}\right]} + \ln {\left[ \frac{T}{\sigma \theta_{rot}}\right]}\ldots\\
%\ldots + &\frac{\theta_{vib}}{T}\left(\frac{1}{e^{\theta_{vib}/T}-1}\right) - \ln{\left[ 1-e^{-\theta_{vib}/T}\right]} + \ln{\left[ g_{e1}\right]}\lrcorner\\
%\frac{p_2}{p_1}&=\left(\frac{V_1}{V_2}\right)^{\gamma}\\
%V_2&=\frac{V_1}{\gamma}\left[\gamma -1 +\frac{p_1}{p_2}\right]\\
%\gamma&=C_p/C_V\\
%A &= U - TS\\
G& = U - TS + pV = H -TS\\ % = A + pV\\
\end{align*}
%dU&=TdS-pdV; \left(\frac{\partial T}{\partial V}\right)_S = -\left(\frac{\partial p}{\partial S}\right)_V\\
%dH&=TdS+Vdp; \left(\frac{\partial T}{\partial p}\right)_S = \left(\frac{\partial V}{\partial S}\right)_p\\
%dA&=-SdT-pdV; \left(\frac{\partial S}{\partial V}\right)_T = \left(\frac{\partial p}{\partial T}\right)_V\\
%dG&=-SdT+Vdp; \left(\frac{\partial S}{\partial p}\right)_T = -\left(\frac{\partial V}{\partial T}\right)_p\\
%&\left(\frac{\partial \left(\Delta G/T\right)}{\partial T}\right)_{p}=-\frac{\Delta H}{T^2}\\
%\mu_{JT}&=\left( \frac{\partial T}{\partial p}\right)_H=\frac{T\left(\frac{\partial V}{\partial T}\right)_p -V}{C_p}\\


\end{multicols*}

\end{document}